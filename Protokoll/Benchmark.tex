\section{Benchmark}
Beim Ausführen von ./main wird ein Benchmark Test durchgeführt. Dieser gibt die jeweils benötigte Zeit aus und erstellt ein .csv für jede Datenstruktur
in den ``result'' folder. 

\subsection{Aufbau}
\label{bench:aufbau}
Es werden zwei files für einen Test benötigt: pre[0-9]+.csv und main[0-9]+.csv. Mithilfe der Einträge im pre-file wird die
Datenstruktur befüllt. Sobald dies abgeschlossen ist, werden alle negativen Werte im main-file aus der Datenstruktur gelöscht 
und alle positiven Einträge hinzugefügt. Somit sind in der Liste nur positive Werte zulässig. Die implementierten Datenstrukturen
würden grundsätzlich aber auch mit negativen Werten und anderen Datentypen funktionieren. Um jedoch einen einfachen 
Benchmark zu realisieren sind nur Einträge zwischen 1 und 2.147.483.646 möglich. 
Nach jeder Entfernung und Hinzufügung eines Wertes wird mithilfe von \textit{contain()} überprüft, ob das Element hinzugefügt bzw.
entfernt wurde. 
Als dritter und letzter Schritt wird mithilfe von \textit{contain()} überprüft, ob sich alle positiven Werte aus dem main-File 
in der Datenstruktur befinden und ob alle negativen Werte aus der Datenstruktur entfernt wurden. 
\\
pre[0-9]+.csv und main[0-9]+.csv sind so aufgebaut, dass im main-File alle Werte aus dem pre-file mit negativen Vorzeichen
vorhanden sind und die positiven Werte nicht im pre-File vorkommen. Desweiteren werden alle Einträge in den beiden Files
zufällig permutiert. 

\subsection{Testcases erstellen}
Die Testcases werden mithilfe eines Python Skriptes erstellt. Dabei ist die Größe der Testcases mithilfe der Konstanten
RAW\_BASE und COOLUMS\_BASE einzustellen. Mit FILE\_AMOUNT kann eingestellt werden, wie viele Testcases erstellt werden.
Dabei werden bei jeder Interation RAW\_BASE und COOLLUM\_BASE mit 2 hoch der aktuellen Iteration-Variable multipliziert. 


\subsection{Interpretation Benchmark Werte}
Die Dauer jedes einzelnen Tests und ob der Test erfolgreich war wird in der Kommandozeile ausgegeben. Desweiteren wird
im Ordner \textit{results} für jede Datenstruktur ein CSV File angelegt. Die Beschreibung der einzelnen 
Features kann aus Tabelle \ref{tab:csv} entnommen werden.
\begin{table}[H]
	%\begingroup

	\setstretch{1.2}
	%\renewcommand{\arraystretch}{1.5} % Default value: 1
	\begin{tabular}{ |m{2.5cm}|  m{13cm}| } 

		\hline
	\multicolumn{2}{|l|}{\textbf{Algemeines}}  \\  \hline
	 Cores: & 		Anzahl der verwendeten Cores    \\ \hline
	 TestSizePre& 	Anzahl der Werte, die im ersten Schritt in die Datenstruktur eingefügt wurden\\ \hline
	 TestSizeMain& 	Anzahl der Werte, die im zweiten Schritt hinzugefügt oder entfernt wurden\\ \hline
	 
	 \multicolumn{2}{l}{} \\ \hline
	 \multicolumn{2}{|l|}{\textbf{Ausschließliches einfügen in die Datenstruktur nur \textit{add()}}}    \\ \hline
	 time write& 	Benötigte Zeit, um die Werte aus dem ersten Schritt in die Datenstruktur einzufügen\\ \hline
	 goToStart write& Anzahl der Vorfälle, die ein erneutes Durchsuchen der Datenstruktur vom Beginn an erfordern\\ \hline
	 lostTime write&  Zeit die durch ein erneutes Durchsuchen verloren geht.\\ \hline
	 \multicolumn{2}{l}{} \\ \hline
	 \multicolumn{2}{|l|}{\textbf{Gemischter Zugriff \textit{add(), remove()} und \textit{contain()}}}    \\ \hline
	 time mixed& 	Benötigte Zeit, um die Werte aus dem zweiten Schritt in die Datenstruktur einzufügen 
	 				bzw. zu entfernen und jeweils anschließend zu überprüfen\\ \hline
	 goToStart mixed& Anzahl der Vorfälle, die ein erneutes Durchsuchen der Datenstruktur vom Beginn an erfordern\\ \hline
	 lostTime mixed& Zeit die durch ein erneutes Durchsuchen verloren geht.\\ \hline
	 \multicolumn{2}{l}{} \\ \hline
	 \multicolumn{2}{|l|}{\textbf{Lesender Zugriff nur \textit{contain()}}}    \\ \hline
	 time check& Benötigte Zeit, um zu überprüfen, ob alle Werte aus dem zweiten Schritt hinzugefügt bzw. entfernt wurden \\ \hline
	 goToStart check& Anzahl der Vorfälle, die ein erneutes Durchsuchen der Datenstruktur vom Beginn an erfordern\\ \hline
	 lostTime check& Zeit die durch ein erneutes Durchsuchen verloren geht.\\ \hline
	\end{tabular} 
	\caption{CSV Spalten}
	\label{tab:csv}
	\end{table}

\subsection{Wiederholungen}
Um ein möglichst genaues Ergebnis zu erhalten, müssen die Benchmarks wiederholt werden. Dazu steht in \textit{benchmark.hpp}
die Konstante REPEAT\_TESTS zur Verfügung. In den CSV-Files werden dann die Durchschnittswerte aller Wiederholungen eingetragen.