\section{Fazit}
Es wurden Unterschiedliche List-baset Sets implementiert und die Performance evaluiert.
Hierbei wurde zwei Kategorien untersucht. Zum einen die Performance bei steigender Listengröße und
die Laufzeit mit unterschiedlicher Anzahl von Cores. 
Es war ersichtlich, dass bei der Verendung von nur einen Core die Liste mit 
einem globalen Lock am schnellsten ist. Schon ab der Verwendung von 2 Threads sind
andere Listen besser. Es hat sich auch gezeigt, dass das Memmorymanagment, also das zurückgeben
von nicht mehr verwendeten Speicher, positive auswirkungen auf die Performance. Ein weiteres
interessantes Resultat war, dass sich die Laufzeit bei der Liste mit \textit{fine-grained locks} mit zwei Threads 
im Vergleich zu einem Thread fast verdreifacht. Es war auch gezeigt, dass Listen bei einer Kollision
mit anderen Threads abbrechen müssen und dadurch wieder beim Listenkopf beginnen müssen. 
Dieses Verhalten kann mit geringen Programmieraufwand erheblich verbessert werden. Hierbei 
gibt es jedoch noch weiteres Verbesserungspotential. \\
Ein weiteres nicht erwartetes Ergebnis war, dass die \textit{lazy} Liste für eine \textit{contain()} Abfrage
erheblich länger benötigt als die \textit{lock-free} benötigt, obwohl die Funktionen fast identisch sind. 
In diesen Projekt sind alle Zugriffe in zufälliger Reihenfolge erfolgt. Eine weitere Möglichkeit, die Performance
zu evaluieren, wäre der Zugriffen welche nicht gleichverteilt sind. Da die Dichte der Zugriffe auf einigen
Stellen der Liste dadurch höher wird, ist davon auszugehen, dass sich die Performance dadurch verändert. 
Dies wurde hier jedoch nicht behandelt und wäre eine Möglichkeit für ein zukünftiges Projekt. 