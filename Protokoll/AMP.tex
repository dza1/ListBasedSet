\documentclass[10pt,a4paper,titlepage,oneside]{article}
\usepackage{LabProtocol}
\usepackage{tikz}
\usepackage{enumitem}
\usepackage{hyperref} 

\usepackage[naustrian]{babel}

\usetikzlibrary{arrows,automata} 

\exercise{}

% enter your data here
\authors{
	Daniel Zainzinger, Matr. Nr. 11777778 \\Noah Bruns, Matr. Nr. 11777778\par
	{\small e11777778@student.tuwien.ac.at\\ e11777778@student.tuwien.ac.at} \par
}


\begin{document}

\maketitle  

\tableofcontents
\newpage
\section{Einleitung}
Für das Programmieren mit mehreren Prozessoren ist es oft erforderlich, eine gemeinsame Datenstruktur zu verwenden. 
Eine möglichkeit ist List based Set. Dabei handelt es sich um eine Liste, bei welcher jedes Element einen Key besitzt.
Dieser Key ist eindeutig und die gesamte Liste ist anhand des Schlüssel sortiert. 
\section{List-based set with fine-grained locks}
\section{List-based set with optimistic synchronization}
\section{List-based set with lazy synchronization}
\section{Lock-free list-based set}


	


		

\end{document}
