\documentclass[10pt,a4paper,titlepage,oneside]{article}
\usepackage{LabProtocol}
\usepackage{tikz}
\usepackage{enumitem}
\usepackage{hyperref} 

\usepackage{wrapfig}
\usepackage{multirow}
\usepackage{tabularx}
\usepackage[naustrian]{babel}

\usetikzlibrary{arrows,automata} 

\exercise{}

% enter your data here
\authors{
	Daniel Zainzinger, Matr. Nr. 11777778 \\Noah Bruns, Matr. Nr. 11777778\par
	{\small e11777778@student.tuwien.ac.at\\ e11777778@student.tuwien.ac.at} \par
}


\begin{document} 

\maketitle  
 
\tableofcontents
\newpage  
\section{Einleitung}
Für das Programmieren mit mehreren Prozessoren ist es oft erforderlich, eine gemeinsame Datenstruktur zu verwenden. 
Eine möglichkeit ist List based Set. Dabei handelt es sich um eine Liste, bei welcher jedes Element einen Key besitzt.
Dieser Key ist eindeutig und die gesamte Liste ist anhand des Schlüssel sortiert. 
Um das Testen und Benchmarken der Datenstrukturen einfach zu halten, werden nur positive Integer Werte in die Datenstruktur eingefügt (siehe \ref{bench:aufbau}).

\section{List-based set with coarse-grained locks}

Bei dieser Implementierung wir vor jedem Methodenaufruf die gesammte Liste gesperrt. Dadurch kann aber die Parallelisierung nicht wirklich ausgenutzt werden, da zu einem Zeitpunkt maximal ein Process auf die Liste zugreifen kann. Daher ähnelt diese Version einer sequentiellen Implementierung.

\subsection{Linearisation Point}

\subsection{Methods:}
\textbf{add}\\
\textbf{contains}\\
\textbf{remove}

\section{List-based set with fine-grained locks}

Im unterschied zu den coarse-grained locks wird bei dieser Implementierung immer nur die derzeitige Node gesprerrt. Beim Iterieren der Liste wird während dem Übergang zur nächsten Node die derzeitige und die darauffolgende Node gesperrt. Der Vorteil gegenüber den coarse-grained locks besteht darin, dass nun wirkliche Parallelisierung möglich ist und verschidene Threads an unterschiedlichen Punkten an der Liste arbeiten können.

\subsection{Linearisation Point}

\subsection{Methods:}
\textbf{add}\\
\textbf{contains}\\
\textbf{remove}

\section{List-based set with optimistic synchronization}

In dieser Implementierung wird beim Suchen eines ELementes zuerst keine Node gesperrt sonder einfach nur eine Iteration durchgeführt. Sobald das gesuchte gefunden wurde wird dieses gesperrt und sichergestellt, dass diese Node immernoch erreichbar ist. Falls dies nicht der Fall sein sollte wird startet der Algorithmus von vorne.

\subsection{Linearisation Point}

\subsection{Methods:}
\textbf{add}\\
\textbf{contains}\\
\textbf{remove}

\section{List-based set with lazy synchronization}

Lazy Synchronization verwendet eine Markierung auf den Nodes die bekannt gibt ob diese Node noch aktiv ist oder nicht. Bei einem Lösch Vorgang wird bei der zu löschenden Node diese Markierung gesetzt. Dadurch wird diese Node bei anderen Methoden, wie zum Beispiel Contains oder Add, ignoriert. 

In späterer folge wird durch das Memory management, ähnlich wie bei optimistic synchronization, die Node gelöscht sobald sichergestellt werden kann, dass kein Process mehr auf die Node zugreift.

\subsection{Linearisation Point}

\subsection{Methods:}
\textbf{add}\\
\textbf{contains}\\
\textbf{remove}

\section{Lock-free list-based set}

\subsection{Linearisation Point}

\subsection{Methods:}
\textbf{add}\\
\textbf{contains}\\
\textbf{remove}



\subsection{Verbesserungen}
In der \textit{find} Funktion wird für das Auslinken eines Nodes wird ein Compare and Swap(CAS) verwendet. Falls dies fehlschlägt, 
weil z.b. der Node zum beispiel bereits ausgelinkt wurde,
wird wieder beim Listenkopf begonnen. Um dieses Verhalten zu verbessern, wurde in \textit{Lock\_free\_impr.cpp} folgendes implementiert:\\
Es wird immer der alte \textit{predecessor} gespeichert. Sollte CAS fehlschlagen, wird überprüft ob der alte \textit{predecessor}
zum auslinken markiert wurde. Ist dies der Fall, werden \textit{predecessor, current} und \textit{successor} auf den jeweiligen Vorgänger
gesetzt und an dieser Stelle weitergemacht. 

\section{Memmory Managment}\label{sec:mem}
Bei \textit{List-based set with optimistic synchronization}, \textit{List-based set with lazy synchronization} und \textit{Lock-free list-based set}
gibt es ein Problem mit den Memmory Managment. Bei diesen Datenstrukturen ist das lesen eines Nodes auch ohne einen Lock möglich. 
Dadurch kann von einen Thread nicht sichergestellt werden, dass nach dem auslinken eines Nodes dieser nicht noch von einen anderen Thread gelesen wird.
\\ 
\textbf{Beispiel:}
\\ Angenommen Thread A liest einen Node und legt sich schlafen. Thread B will diesen Thread löschen. Dazu linkt er den Node aus der Datenstruktur aus
. Anschließend wird dieser von Thread B gelöscht. Wenn nun Thread A wieder aufwacht, und auf den Node zugreiffein will, ist der Speicher bereits freigegeben
und es tritt ein Segfault auf. 

\subsection{Markierung beim Lesen}
Als erster Ansatz wurde versucht, mithilfe eines Counters einen Node zu markieren, dass dieser gerade gelesen wird. Dieser Ansatz ist jedoch nicht 
Zielführend und ist nur zum verdeutlichen der Schwierigkeit dieses Problems erwähnt. \\
Bevor ein Thread einen Node liest, erhöht er einen Counter in diesen Node. Nachdem ein Node verlassen wurde, wird dieser Counter wieder verringert.
Ein Thread, der diesen Node löschen will, darf dies erst tun, sobald der Node ausgelinkt und der Counter den Wert 0 hat. \\
Damit ein Thread den nächsten Node markieren kann, muss er zuerst den ``next'' Pointer aus den aktuellen Node auslesen und dann den Counter erhöhen (next$\rightarrow$counter++).
Es ist jedoch nicht möglich, das Auslesen des next Pointers und das erhöhen des Counters atomar auszuführen. Auch ein ``compare and swap'' ist hierbei nicht hilfreich,
da das vergleichen mit einen freigegebenen Speicherbereich zu einen Segmentation Fault führt. 


\subsection{Delete Queue mit Zeitstempel}
Um zu vermeiden, dass ein Node gelöscht wird, während er von einen anderen Thread noch gelesen wird, muss sichergestellt werden, dass dieser ``frei'' ist. 
Dazu wurden Zeitstempel eingeführt. Jeder Thread besitzt einen globalen Integer, wo sein Zeitstempel gespeichert wird. Sobald eine Operation
wie Beispielsweise ``add'' abgeschlossen ist, wird dieser Zeitstempel erhöht.  Will nun ein Thread einen Node löschen, wird dieser zuerst ausgelinkt. 
Anschließend werden alle Zeitstempel ausgelesen und gemeinsam mit einen Pointer auf den zu löschenden Node in einer Queue hinzugefügt. Diese 
Queue wird lokal im Thread gespeichert und wurde als C++ Vektor realisiert. Nach den Abschluss einer Operation wie Beispielsweise ``add'', wird 
die Queue mit zu löschenden Kandidaten durchlaufen. Für jeden Node werden nun die gespeicherten Zeitstempel mit den aktuellen Zeitstempel verglichen. 
Wenn sich alle Zeitstempel eines Nodes von den aktuellen Zeitstempel unterscheiden, ist sichergestellt, dass dieser Node von keinen Thread mehr gelesen wird. 
Nun kann der Speicher des Nodes wieder Freigegeben werden und aus der Queue entfernt werden. 
\\Bevor ein Thread geschlossen wird, muss sichergestellt werden
dass die Queue gelehrt wird und somit der Speicher aller Kandidaten gelöscht wird. Da hierbei die Zeitstempel nicht mehr überprüft werden,
kann die Liste nach dem Schließen eines Threads nicht mehr verwendet werden.  
\section{Benchmark}
Beim Ausführen von ./main wird ein Benchmark Test durchgeführt. Dieser gibt die jeweils benötigte Zeit aus und erstellt ein .csv für jede Datenstruktur
in den ``result'' folder. 

\subsection{Aufbau}
\label{bench:aufbau}
Es werden zwei files für einen Test benötigt: pre[0-9]+.csv und main[0-9]+.csv. Mithilfe der Einträge im pre-file wird die
Datenstruktur befüllt. Sobald dies abgeschlossen ist, werden alle negativen Werte im main-file aus der Datenstruktur gelöscht 
und alle positiven Einträge hinzugefügt. Somit sind in der Liste nur positive Werte zulässig. Die implementierten Datenstrukturen
würden grundsätzlich aber auch mit negativen Werten und anderen Datentypen funktionieren. Um jedoch einen einfachen 
Benchmark zu realisieren sind nur Einträge zwischen 1 und 2.147.483.646 möglich. 
Nach jeder Entfernung und Hinzufügung eines Wertes wird mithilfe von \textit{contain()} überprüft, ob das Element hinzugefügt bzw.
entfernt wurde. 
Als dritter und letzter Schritt wird mithilfe von \textit{contain()} überprüft, ob sich alle positiven Werte aus dem main-File 
in der Datenstruktur befinden und ob alle negativen Werte aus der Datenstruktur entfernt wurden. 
\\
pre[0-9]+.csv und main[0-9]+.csv sind so aufgebaut, dass im main-File alle Werte aus dem pre-file mit negativen Vorzeichen
vorhanden sind und die positiven Werte nicht im pre-File vorkommen. Desweiteren werden alle Einträge in den beiden Files
zufällig permutiert. 

\subsection{Testcases erstellen}
Die Testcases werden mithilfe eines Python Skriptes erstellt. Dabei ist die Größe der Testcases mithilfe der Konstanten
RAW\_BASE und COOLUMS\_BASE einzustellen. Mit FILE\_AMOUNT kann eingestellt werden, wie viele Testcases erstellt werden.
Dabei werden bei jeder Interation RAW\_BASE und COOLLUM\_BASE mit 2 hoch der aktuellen Iteration-Variable multipliziert. 


\subsection{Interpretation Benchmark Werte}
Die Dauer jedes einzelnen Tests und ob der Test erfolgreich war wird in der Kommandozeile ausgegeben. Desweiteren wird
im Ordner \textit{results} für jede Datenstruktur ein CSV File angelegt. Die Beschreibung der einzelnen 
Features kann aus Tabelle \ref{tab:csv} entnommen werden.
\begin{table}[H]
	%\begingroup

	\setstretch{1.2}
	%\renewcommand{\arraystretch}{1.5} % Default value: 1
	\begin{tabular}{ |m{2.5cm}|  m{13cm}| } 

		\hline
	\multicolumn{2}{|l|}{\textbf{Algemeines}}  \\  \hline
	 Cores: & 		Anzahl der verwendeten Cores    \\ \hline
	 TestSizePre& 	Anzahl der Werte, die im ersten Schritt in die Datenstruktur eingefügt wurden\\ \hline
	 TestSizeMain& 	Anzahl der Werte, die im zweiten Schritt hinzugefügt oder entfernt wurden\\ \hline
	 
	 \multicolumn{2}{l}{} \\ \hline
	 \multicolumn{2}{|l|}{\textbf{Ausschließliches einfügen in die Datenstruktur nur \textit{add()}}}    \\ \hline
	 time write& 	Benötigte Zeit, um die Werte aus dem ersten Schritt in die Datenstruktur einzufügen\\ \hline
	 goToStart write& Anzahl der Vorfälle, die ein erneutes Durchsuchen der Datenstruktur vom Beginn an erfordern\\ \hline
	 lostTime write&  Zeit die durch ein erneutes Durchsuchen verloren geht.\\ \hline
	 \multicolumn{2}{l}{} \\ \hline
	 \multicolumn{2}{|l|}{\textbf{Gemischter Zugriff \textit{add(), remove()} und \textit{contain()}}}    \\ \hline
	 time mixed& 	Benötigte Zeit, um die Werte aus dem zweiten Schritt in die Datenstruktur einzufügen 
	 				bzw. zu entfernen und jeweils anschließend zu überprüfen\\ \hline
	 goToStart mixed& Anzahl der Vorfälle, die ein erneutes Durchsuchen der Datenstruktur vom Beginn an erfordern\\ \hline
	 lostTime mixed& Zeit die durch ein erneutes Durchsuchen verloren geht.\\ \hline
	 \multicolumn{2}{l}{} \\ \hline
	 \multicolumn{2}{|l|}{\textbf{Lesender Zugriff nur \textit{contain()}}}    \\ \hline
	 time check& Benötigte Zeit, um zu überprüfen, ob alle Werte aus dem zweiten Schritt hinzugefügt bzw. entfernt wurden \\ \hline
	 goToStart check& Anzahl der Vorfälle, die ein erneutes Durchsuchen der Datenstruktur vom Beginn an erfordern\\ \hline
	 lostTime check& Zeit die durch ein erneutes Durchsuchen verloren geht.\\ \hline
	\end{tabular} 
	\caption{CSV Spalten}
	\label{tab:csv}
	\end{table}

\subsection{Wiederholungen}
Um ein möglichst genaues Ergebnis zu erhalten, müssen die Benchmarks wiederholt werden. Dazu steht in \textit{benchmark.hpp}
die Konstante REPEAT\_TESTS zur Verfügung. In den CSV-Files werden dann die Durchschnittswerte aller Wiederholungen eingetragen. 

	


		

\end{document}
